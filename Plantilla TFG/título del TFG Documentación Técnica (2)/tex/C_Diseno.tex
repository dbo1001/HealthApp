\apendice{Especificación de diseño}

\section{Introducción}
En este capitulo veremos como todo lo relacionado con el diseño de la herramienta, por lo general ha llevado un pequeño estudio detrás para buscar la mejor ergonomía posible además del correcto desarrollo.
\section{Diseño de datos}
El diseño tanto visual como de los datos, se decidió antes de empezar a realizar el proyecto, por lo general se basa en:
\begin{itemize}
\item Diseño visual organizado en Frames y ordenado por packs, sistema proporcionado por Tkinter, que permite subdividir cada pantalla en pequeñas proporciones y estas en sub-proporciones y así sucesivamente, pudiendo colarcarla a grandes rasgos (arriba, abajo,izquierda, derecha, ocupando parte, todo el eje x, etcétera.).
\item Alimentos tratados según el estándar de medida para la información nutricional impuesto en todo el mundo.
\item Manejo de Bases de datos en archivos ".xlsx"
\item Manejo interno de los datos a través de DataFrames
\end{itemize}
\section{Diseño procedimental}

\section{Diseño arquitectónico}


