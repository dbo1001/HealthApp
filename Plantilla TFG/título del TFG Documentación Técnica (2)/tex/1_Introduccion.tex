\capitulo{1}{Introducción}

Este proyecto, desarrollado en Python, nace de la idea de una aplicación nueva de enseñanza dieto-terapéutica. Este trabajo, en cierto modo, hace honor al propio nombre de " informática ", llevando a los usuarios una información, que como se demuestran en diversos estudios citados y desarrollados mas adelante en esta memoria, la mayoría de gente desconoce; y automatizando de manera ágil y transparente al usuario el aprendizaje exhaustivo sobre la dieto-terapia, para conseguir que el usuario consiga un habito de vida saludable y no reciba una estricta dieta imposible de seguir.
Actualmente se estima que alrededor del 81 por ciento de los españoles, que se proponen cambiar de hábitos y seguir una dieta acaba desistiendo. Eso en su mayoría nace de la dificultad, de mantener un ritmo constante de rigor a la hora de seguir una serie de instrucciones que de una manera u otra te condicionan, impidiéndote hacer una vida 100 por ciento, más allá de lo que en el ámbito alimenticio se refiere, como es en el ámbito social.
Por ello, Combinando el análisis y tratamiento de datos, mediante sistemas informáticos; con los últimos estudios relacionados con la dieto-terapia. Se busca crear una aplicación, que más que inculcar una dieta al usuario, sirva de guía para percibir un nuevo estilo de vida como propio. Gracias a esto, se puede llegar a conseguir un gran avance medico/informático, debido a la importancia que la dieto-terapia tiene en la salud de las personas.
Se calcula que una de cada cinco muertes en el mundo esta relacionada con la mala alimentación, por supuesto, dejando fuera las muertes por desnutrición. Por lo tanto, el proyecto es sustenta, de la constante necesidad y la creciente demanda de sistemas que ayuden y faciliten a una adecuada alimentación o dieto-terapia para el cuidado personal. Se pensó en un modelo de auto-aprendizaje, debido a que se ha considerado que pese a ser más lento es más sencillo para el ser humano adecuar un estilo de vida, que no seguir una dieta estricta. La idea principal es que el usuario, en base a sus necesidades, elija de manera libre el menú que vaya a consumir en cada plato a lo largo del día, teniendo en cuenta en todo momento, como de bueno o mala es esa decisión para él.
" el estilo de aprendizaje auto-aprendizaje es mas efectivo cuando reconoces que es lo mas importante para ti ”,  (estudio guardado en barra marcadores), Tras estudiar varios métodos de aprendizaje, se consideró, que debido a la relación directa: dieta/salud, el auto-aprendizaje sería el mas oportuno, de esta manera a un ritmo lento y constante, el usuario aprende a comer bien, y no se pierde entre dietas.
Se pensaron en una serie de pasos imprescindibles para inculcar un nuevo estilo de vida:

\begin{enumerate}
\item	Identificar que se quiere cambiar
\item	Metas específicas, realistas y constantes
\item	Creación de un plan
\item	Recordatorios que seguir
\item	Mide tus avances
\end{enumerate}
Se ha buscado que la herramienta desarrollada durante todos estos meses, de una sensación de falsa simplicidad, es decir, se ha buscado hacer lo mas simple, para abordar cuestiones complejas, de manera común, sin rizar el rizo, pero evitando dejar el mínimo de cabos sueltos, se ha buscado cumplir de manera pasiva cada uno de estos puntos dentro de la aplicación, puesto que dentro de sus objetivos y funcionalidades esta:

\begin{enumerate}
\item	Su función principal, aprender, y cambiar el estilo de vida a través de la dietoterapia, cumpliendo el objetivo uno, de identificar que parte de tu estilo de vida, deseas cambiar.
\item	Posiblemente, una de las partes mas complicadas e invisibles del proyecto. El proyecto siempre va a recomendar lo que considera la mejor opción, y siempre va a ser en base a los datos del cliente, consiguiendo, que las metas sean constantes, y que poco a poco se vayan volviendo mas estrictas, y pase de manera imperceptible al usuario.
\item	Creación de un plan, la mas fácil e importante de los cinco pasos. Si se piensa detenidamente, es el usuario el que se crea su propio plan, pero siempre teniendo una “mano invisible”, que le ayudo a redirigir sus opciones, se decir, cada cosa que decide el usuario dentro de la aplicación queda registrada, y se ve en los gráficos sobre la calidad, como en verdad ha comido nuestro cliente.
\item Igual es la menos intuitiva para el usuario, pero no nos olvidamos de ella, se encuentra en el historial, esos gráficos que permiten ver al usuario como avanza y en la barra que le dice como esta comiendo en el día. Dentro de estas opciones aparecen reflejados los puntos 4 y 5.
\end{enumerate}

A lo largo de esta memoria será plausible el como se ha desarrollado cada parte del proyecto, tanto en el ámbito informático como en el nutricional. Detallando cada estudio, desarrollo y opción que se ha tenido a lo largo de este, para que toda decisión tomada a lo largo del periodo de desarrollo quede clara, y se entienda el ¿por qué?, de cada decisión. Con este proyecto se consigue proyectar la unión de dos mundos complejos, de la manera más simple posible, haciendo alarde de que en la sencillez se oculta la complejidad de todo esto.
