\apendice{Plan de Proyecto Software}

\section{Introducción}
En este Anexo se detallarán las distintas etapas, fases, cambios y procesos que ha pasado el programa hasta finalizar su desarrollo. También dentro de este proceso hay que tener en cuenta la viabilidad de llevar a cabo el proyecto hasta una versión comercial.
\\
Utilizando el marco de trabajo SCRUM \cite{scrum}, para metodología ágil. Se lleva a cabo una evolución estructurada del proyecto. Sabiendo en la medida de lo posible cual iba a ser el avance de este.
\\

\section{Planificación temporal}
El proyecto fue dividido en diferentes etapas para lograr un desarrollo constante:
\begin{itemize}
\item Nacimiento y pulimento de la idea
\item Desarrollo:
\begin{itemize}
\item Creación de la estructura de los datos.
\item Carga y manejo de los datos
\item Creación del proyecto por interfaz de comando
\item Traducción del proyecto a interfaz gráfica
\item Pulimento de detalles y nuevas funcionalidades
\end{itemize}
\item Pruebas
\item Desarrollo de la documentación pertinente
\end{itemize}
Planificación principal del proyecto. Sostenida sobre un cronograma para cumplir las metas pautadas.\\
Cronograma:

\imagen{cronograma}{Cronograma del proyecto completo}
\subsection{Fases}
\textbf{\textsc{Perfeccionamiento de la idea}}\\
La primera fase del proyecto fue elegir y perfeccionar una idea propia.\\

Se llevó a cabo un estudio sobre la demanda actual de los usuarios junto a las ultimas investigaciones realizadas, sumadas al interés del alumno por la ingeniería biomedica. Tras varios días de estudio se llegó a la conclusión de que una aplicación que no impartiese una dieta al usuario sino que ayudase al usuario a llegar por sus propios métodos al fin establecido, parecía un reto en cuanto investigación y avance.\\

\textbf{\textsc{Definir Objetivos}}\\
Tras la primera reunión, se pautaron los diferentes objetivos. Pautando la meta y requisitos que la aplicación debía cumplir con el paso del tiempo y hasta su etapa final.\\

\textbf{\textsc{Determinar los requisitos}}\\
En la reunión para hablar de los objetivos, se desarrollaron los requisitos. Ambas partes están relacionadas. Naciendo requisitos del sistema, según el alcance de los objetivos.\\

\textbf{\textsc{Aplicación en Shell}}\\
Se desarrollaron los algoritmos y las funcionalidades principales, con interfaz de linea de comandos. Permitió un desarrollo completo y centrado en el perfeccionamiento de dichas funciones.\\

\textbf{\textsc{Pruebas del software}}\\
Antes de avanzar con el desarrollo de la aplicación, se realizaron una serie de pruebas. Se corrigieron los fallos y errores resultantes durante dichas pruebas. Una vez terminados los errores, se pasó a la siguiente fase.\\

\textbf{\textsc{Completar Desarrollo}}\\
Construcción del programa en el resto de ámbitos, proporcionando al usuario una interfaz visual clara y simple, y dando las funcionalidades posibles de dicha interfaz al usuario como son:  los gráficos, la persistencia y la navegabilidad.\\

\textbf{\textsc{Pruebas del software finales }}\\
Se llevaron a cabo numerosas pruebas del sistema. Comprobando su solidez, y consistencia ante posibles errores. Dando por finalizada la aplicación.
\section{Viabilidad}
Es necesario que el proyecto sea viable en numerosos ámbitos, tanto en el económico como en el legal, además de una viabilidad consciente a la hora del desarrollo futuro.
\subsection{Viabilidad económica}
En este apartado se realizará un análisis de la viabilidad económica usando algunas de las técnicas más utilizadas en iniciativas emprendedoras.\\
Este apartado se divide en varias secciones, diferencia dos tipos de análisis:
\begin{enumerate}
\item Análisis  de la viabilidad empresarial
\item Análisis económico de la viabilidad del proyecto, y margen de beneficios.
\end{enumerate}
\subsubsection{Análisis de la viabilidad empresarial}
\textbf{\textsc{Análisis DAFO}}\\
El análisis DAFO \cite{DAFO} es una herramienta de estudio de la situación de un proyecto, en base a sus principales características.
\begin{itemize}
\item Oportunidades: Crear algo nuevo y ser el primero en crearlo, siempre opta a ser el cabecilla del mercado
\item Fortalezas: Producto innovador que se crea en la cresta de la ola que es la moda por la salud y el fitness
\item Amenazas: El mundo de las aplicaciones es un mar lleno de rivales, donde malas aplicaciones triunfan y buenas fallan
\item Debilidades: La dificultad de entrada al mercado al ser una aplicación nueva	
\end{itemize}
\textbf{\textsc{Matriz MCKINSEY}}\\
La matriz mckinsey \cite{matriz} es una herramienta de marketin estratégico que ayuda a la empresa a tomar decisiones sobre la inversión en el proyecto. Ayuda a priorizar la inversión, o llevar una estrategia mas conservadora
\imagen{MCKENSEY}{Matriz Mckensey del plan de negocio}
\imagen{ALLADODELAMCKENSEY}{Análisis del resultado de la matriz Mckensey}
\pagebreak
\subsubsection{Análisis económico}
La formula para calcular la Amortización de los diferentes elementos utilizados a lo largo del programa es:

\begin{equation}
Amortización = \frac{(Valor)}{(VidaUtil)}
\end{equation}

\textbf{\textsc{Costes Hardware:}}\\S
Durante el desarrollo del proyecto únicamente ha sido necesario utilizar un ordenador y un router con internet.
\begin{itemize}
\item Ordenador portátil: 460euros \\Vida útil de aproximadamente 5 años.\\460euros/(12*5) meses = 7,66euros/mes.\\Durante 7 meses de trabajo = 7,66*7 = 53.66 euros brutos.
\item Router: 50euros. \\ 50euros/(12*5)meses)*7 meses de trabajo = 5,82 euros brutos.
\end{itemize}


\textbf{\textsc{Costes Software:}}\\
La única licencia software de pago usada durante el desarrollo del proyecto ha sido Windows 10 Home, la cual tiene un coste de 120 euros.\\
Se estimamos su vida útil en 4 años \cite{vidautil}. Por lo tanto:\\
120euros/(12*4)meses = 1,5euros/mes.\\
Por 7 meses trabajados son 17,5 euros brutos.\\

\textbf{\textsc{Costes Materiales:}}\\
Diferentes costes ajenos al software o hardware necesario para la integra realización del proyecto.
\begin{itemize}
\item Mobiliario - Oficina CEEI Burgos, con luces e inmuebles: 200euros/mes.\\
Durante 7 meses de trabajo = 200euros/mes*7 = 1.400 euros brutos\\
\item Cartulinas - 2 euros
\item Libretas - 5euros
\item Pizarra - 10euros
\end{itemize}

\textbf{\textsc{Costes de totales:}}\\
Costes del trabajador según el régimen general de la Seguridad Social en 2019. \cite{ssocial} \\
Porcentajes de cotización:
\begin{itemize}
\item Comunes: 23.6\%
\item Desempleo: 5.5\%
\item Formación profesional: 0.7\%
\item Total acumulado: 29,8\%
\end{itemize}
A continuación se mostrarás las diferentes cotizaciones respecto a la situación del alumno.\\
Coste real del trabajador: 4070*1,298 =  5.282euros.\\
Coste total del desarrollo del proyecto:\\
5282+2+5+10+17.5+5.82+1400+53.66 = 6775.98euros brutos\\

\textbf{\textsc{Beneficios:}}\\
En este aparatado se estima los beneficios que se pueden obtener del proyecto, y la amortización en años, en base al número y precio de ventas.\\
Supongamos que esta herramienta, es usada de manera privada por usuarios, y diferentes clínicas destinadas al reparto entre sus clientes.\\
Dos posibles paquetes de venta:
\begin{itemize}
\item 1 licencia X 130euros.
\item 10 licencias X 1.000euros.
\end{itemize}

Suponiendo la venta a 10 personas para uso personal, y 3 paquetes de diez unidades para empresas sanitarias por año:\\
10 unidades * 130euros = 1.300euros\\
3 Packs de 10 unidades * 1000euros = 3000euros\\
Lo que saldría de media a: 4.300euros/año\\
Recuperando la inversión en: 6775.98euros/4300eurosXaño = 1.53 años.\\
A partir de ese periodo, aproximadamente año y medio, empezaríamos a tener beneficios.

\subsection{Viabilidad legal}
\subsubsection{Licencias}
\textbf{Licencias de las librerías}\\
\begin{itemize}
\item MatplotLib, licencia BSD.
\item Numpy, licencia BSD.
\item Pandas, licencia BSD.
\item Auto-py-to-exe, licencia MIT.
\item webbrowser, open source.
\item os-win, licencia Apache software.
\item Pillow, licencia \textit{Open source Pil Software license}.
\item openpyxl, licencia MIT. 
\end{itemize}
La licencia BSD, es una licencia software libre permisiva, utilizada para los sistemas operativos BSD (Berkeley Software Distribution.\\
La licencia MIT, es una licencia de software libre permisiva lo que significa que apenas tiene limitaciones en la reutilización de sus recursos.\\
Según la guía de licencias de Python \cite{licencias}, tanto la licencia BSD como MIT, son adecuadas para este proyecto. No obstante, debido a que la licencia MIT es menos restrictiva, será la usada para el desarrollo. Con la licencia BSD, se ha de citar al creador del software usado y al propio software.
La licencia Mit \cite{mit}
\begin{enumerate}
\item \textbf{Condiciones:}  La condición es que la nota de copyright y la parte de los derechos se incluya en todas las copias o partes sustanciales del Software. Esta es la condición que invalidaría la licencia en caso de no cumplirse.
\item \textbf{Derechos:}  sin restricciones; incluyendo usar, copiar, modificar, integrar con otro Software, publicar, sublicenciar o vender copias del Software, y además permitir a las personas a las que se les entregue el Software hacer lo mismo.
\item \textbf{Limitación de responsabilidad::}  finalmente se tiene un disclaimer o nota de limitación de la responsabilidad habitual en este tipo de licencias.
\end{enumerate}
\subsubsection{Sociedad Limitada}
Se consideraría una sociedad limitada, al ser el alumno un autónomo presentando el proyecto para lo que sería su propia empresa con la serie de inversiones que esto conlleva, y la viabilidad legal-económica que trae consigo.
Siendo al alumno autónomo y considerando este proyecto parte de su empresa. Se considera el proyecto como una sociedad limitada.\\

\textbf{\textsc{Forma jurídica}}\\
Descripción de la forma jurídica\\
\\
Elección de una sociedad limitada por los motivos que me benefician:\\
- Capital Social mínimo: 3.000 euros  \\

- Nº socios mínimo:1 \\
- 1socio 100\% del Capital Social.\\
- Limitacion a los bienes de la empresa en caso de deudas o cierre \\


\textbf{\textsc{Trámites para la constitución de la empresa}}\\
Descripción en detalle de los trámites de constitución\\
Solicitar el Certificado de Denominación Social.\\

El primer paso hay que dar es obtener el Certificado de Denominación Social. Dicho certificado acredita que no existe ninguna otra sociedad ya constituida que tenga la misma denominación social que la que pretendemos constituir y debe incorporarse a la escritura de constitución.
Se obtiene en el Registro Mercantil Central, que tiene su sede en Madrid, y se podrá solicitar el certificado respecto de una o varias denominaciones sociales hasta un máximo de cinco. El certificado debe ir a nombre de cualquiera de las personas físicas o jurídicas que van a constituir la sociedad limitada como socios fundadores de la misma.\\

El Registro Mercantil Central deberá expedirlo de forma telemática en el plazo máximo de 1 día hábil contado desde la fecha de la solicitud.
El Notario está obligado a solicitarlo por vía telemática, a menos que los interesados manifiesten su intención de solicitarlo ellos mismos.
En el caso de que sean los propios interesados los que lo soliciten, pueden hacerlo de forma presencial rellenando el impreso oficial que deberán presentar en dicho Registro, o bien solicitarlo “on line” a través de la página web del Registro 
Una vez obtenido, el certificado caduca a los dos meses de su fecha. Esto no obstante, se pueden solicitar nuevas certificaciones de la misma denominación social si caducan las anteriores, dado que la denominación social queda reservada a favor del solicitante durante el plazo de quince meses.
Firmar la Escritura Pública de Constitución.\\

La sociedad de responsabilidad limitada se constituye mediante Escritura Pública otorgada ante Notario por la totalidad de los socios fundadores.\\
\\
\textbf{Firmar la Escritura Pública de Constitución.}
La sociedad de responsabilidad limitada se constituye mediante Escritura Pública otorgada ante Notario por la totalidad de los socios fundadores.
\\
\\
La escritura de constitución debe contener:
\begin{enumerate}
\item La identidad de los socios.
\item La voluntad de constituir una sociedad limitada.
\item La aportación de cada socio y las participaciones asignadas en pago de su aportación.
\item Los estatutos de la sociedad.
\item El sistema de administración que inicialmente se establezca para la sociedad.
\item La identidad de la persona que inicialmente se encargue de la administración y de la representación de la sociedad.
Los socios deben elaborar unos Estatutos Sociales que se incorporarán a la Escritura de Constitución y por los que se regirá la sociedad. Dicha tarea
\end{enumerate}
pueden encargarla los socios al Notario que autorizará la escritura de constitución.\\
Los Estatutos deben contener las siguientes menciones:
\begin{enumerate}
\item Denominación de la sociedad, en la que deberá figurar necesariamente la expresión “sociedad de responsabilidad limitada”, “sociedad limitada” o sus abreviaturas “S.R.L.” o “S.L.”.
\item Objeto social, que es la actividad a la que se va a dedicar la sociedad.
\item Fecha de cierre de cada ejercicio social.
\item Domicilio social dentro del territorio español.
\item Capital social, participaciones en que se divida, valor nominal de cada participación y numeración de las mismas.
\item Sistema de administración de la sociedad.
\end{enumerate}

Si los socios realizan aportaciones dinerarias a la sociedad, deberán entregar al Notario un certificado que acredite el depósito en una Entidad de Crédito a nombre de la sociedad de las cantidades aportadas por los socios. La fecha del depósito bancario no podrá ser anterior en más de dos meses a la fecha en la que se firme la escritura de constitución de la sociedad. También cabe la posibilidad de que los socios entreguen directamente al Notario el dinero en que consista su aportación a la sociedad para que sea el propio Notario el que constituya el depósito en el plazo de cinco días hábiles.\\
Si se trata de aportaciones no dinerarias (inmuebles, maquinaria, vehículos, etc), los socios deberá entregar al Notario los títulos de propiedad de tales bienes o la documentación relativa a los mismos.\\

La sociedad puede dar comienzo a sus operaciones comerciales desde la fecha en que se otorga la Escritura de Constitución, aunque no esté inscrita aún en el Registro Mercantil, salvo que en la propia escritura se haya fijado una fecha posterior para el comienzo de las operaciones de la sociedad.\\

\textbf{El Impuesto de Operaciones Societarias.}\\
Con anterioridad, la constitución de una sociedad limitada generaba para la misma la obligación de pagar el Impuesto de Operaciones Societarias al tipo del uno por ciento (1\%) del capital de la sociedad.\\

En la actualidad, a partir del Real Decreto-Ley 13/2010, la constitución de una sociedad limitada está exenta del pago del Impuesto de Operaciones Societarias. Además, para inscribirla en el Registro Mercantil, no será necesaria la presentación en dicho Registro del impreso de autoliquidación en el que se alegue la exención.\\

\textbf{Solicitud del N.I.F. provisional.}\\
El mismo día de la firma de la escritura, el Notario solicitará tambiéntelemáticamente un N.I.F. provisional para la sociedad, que se convertirá en definitivo cuando la sociedad se inscriba en el Registro Mercantil correspondiente. Una vez convertido en definitivo el .N.I.F asignado inicialmente, Hacienda deberá comunicarlo por vía telemática al Notario y al propio Registro Mercantil donde está inscrita la sociedad.\\

\textbf{La Inscripción en el Registro Mercantil.}
La Escritura de Constitución otorgada ante Notario debe inscribirse obligatoriamente y con carácter constitutivo en el Registro Mercantil de la provincia correspondiente al domicilio de la sociedad. A estos efectos, el Notario debe remitir telemáticamente una copia de la escritura de constitución al Registro correspondiente, a menos que los interesados soliciten lo contrario. El Registrador Mercantil tiene un plazo máximo de 15 días para inscribir la sociedad.\\

Una vez inscrita, la sociedad adquiere su personalidad jurídica como sociedad de responsabilidad limitada.
Es necesario también publicar la inscripción en el B.O.R.M.E., cuyas tasas se pagarán telemáticamente.\\

Finalmente, para iniciar la actividad del negocio deberán cumplirse otros trámites  ante la Hacienda Pública como dar de alta a la sociedad en el Impuesto de Actividades Económicas (I.A.E.), salvo que se trate de sociedades exentas, o realizar la declaración censal, que es el alta de la sociedad a los efectos del I.V.A.\\

Igualmente será preciso en la mayoría de los casos la obtención de lalicencia de apertura del establecimiento de la empresa y la licencia de obras, en el caso de que se efectúen obras de reforma en el local en el que se va a realizar la actividad. Ambas licencias deberán tramitarse ante el Ayuntamiento del municipio en el que se encuentra el local de la sociedad.\\
También deberá tramitarse el alta la empresa en la Seguridad Social, que se efectúa en la Tesorería de la misma, y la comunicación de apertura del centro de trabajo, que se realiza en el Ministerio de Trabajo.\\


