\capitulo{7}{Conclusiones y Líneas de trabajo futuras}

A continuación, se desarrollarán las diferentes conclusiones surgidas a lo largo de la creación del proyecto, a la par, de las líneas futuras de desarrollo que el proyecto a de seguir para la explotación a nivel tanto empresarial como informático.
\\
\section{Conclusiones}
A continuación, se desarrollarán la conclusión que el proyecto ha ido dejando a lo largo de su creación:
\begin{itemize}
\item	La dificultad de inculcar, enseñar o mostrar un camino a un estilo de vida más saludable a través de una aplicación para ordenador. En principio no se tienen ningún tipo de información, ni los recursos necesarios en cuanto a métodos pedagógicos, ni bases de datos que lo contemplen. Por lo que se partió de cero, basándonos en diseños, parcialmente similares, pero sin asemejarse demasiado a lo necesitado. Resulta un reto conseguir crear una aplicación que sirva para el aprendizaje o la enseñanza de un campo concreto.
\item	A lo largo de los años de estudio del alumno, se han realizado múltiples programas, o funciones, pero siempre eran cosas que: o ya existían y podías sacar información de diversas fuentes, o el profesor ponía al servicio del alumno, las herramientas necesarias para hacerlo. El verse envuelto en un escenario donde se debía crear de cero cosas que no existían, fue un auténtico reto. La idea de estructurar primero una función o algoritmo, para ir desarrollándolo, cubriendo cualquier imperfección obviada durante el proceso, ayudo a valorar las facilidades que son otorgadas durante los años de estudio.
\item	El análisis de datos desde una hoja de cálculo. Se escogió una hoja de cálculo como base de datos, para facilitar el manejo y trato de los datos. A la larga acabó generando numerosas dificultades, generando diferentes problemas que no hubieran aparecido en casa de usar una base de datos real.
\item	Fue complicado la idea de desarrollar un proyecto desde cero, a lo largo de los estudios del alumno, se han realizado varios programas de distinta índole, pero siempre se partía de una base dada por el profesor y un guion a seguir. La idea de crear un proyecto, y tener que ir desarrollando y creando una serie de objetivos, fue complicado, pues muchas veces eran objetivos absurdos, imposibles, o que se desviaban totalmente del sino del proyecto.
\item	Crear un diseño desde cero, que fuera intuitivo, y de fácil navegabilidad fue un auténtico reto, teniendo que variar en ocasiones el diseño para mejorar la funcionalidad.
\item	Las limitaciones de las interfaces gráficas en Python. Como se ha nombrado en diversas ocasiones, las limitaciones que tienen dichas interfaces gráficas, y particularmente la librería Tkinter, ha resultado un gran inconveniente en varias etapas del proceso.
\item	El trabajo de investigación también fue complejo y extenso, sobre todo en el ámbito más médico-alimenticio donde existen diversas fuentes de información de las cuales muy pocas son fiables. Por ello se optó por una vía más personal como es la búsqueda de profesionales en la materia que pudiera apoyar los datos.
\end{itemize}

\section{Lineas de trabajo futuras}
A continuación, se plasmarán las posibles mejoras y adaptaciones del programa en caso de seguir con su desarrollo hasta llegar a una etapa final. Debido a la situación del proyecto enfocado, como un programa altamente funcional, del cual se espera un futuro desarrollo, no solo a nivel de aplicación, sino a nivel de proyecto empresarial; Se explicarán diferentes campos de ampliación del proyecto.
\subsection{Separación modular y MVC}
Pese a que la herramienta desarrollada, tiene su propio sistema organizativo, variante del modelo-vista-controlador, no deja de ser un método único, usado exclusivamente durante el desarrollo de este programa, por lo que existe una falta de estandarización, donde cualquier programador que quiera continuar el proyecto. Deberá estudiar el método de trabajo. Además, pese a que el programa se separa en varios módulos, debería ser posible separarlo en más módulos diferentes, más específicos y no exclusivamente en los tres módulos principales usados en el proyecto (Sin contar el Main).
\subsection{Base de datos en Servidor}
La idea de una aplicación con una amplia base de datos de alimentos, y una serie de datos sobre los usuarios y sus respectivas patologías, no puede ser tratada de manera local. Debido a la ley de protección de datos \cite{LeyProteccionDatos}, los datos de carácter médico, son datos muy restrictivos y de alta importancia, penados con cárcel.\\
Por ello se implantaría una base de datos en un servidor On-line altamente protegido para evitar cualquier tipo de problemas. Además de lograr la expansibilidad de la base de datos.\\
Se parte de una pequeña muestra, de todos los posibles menús y patología que se podrían añadir a la base de datos. No deja de ser un Start-Up de un gran desarrollo, con los ejemplos básicos para la muestra de su funcionamiento.

\subsection{Perfeccionamiento del algoritmo de recomendación}
Pese a ser el eje central de la aplicación es un algoritmo todavía a perfeccionar. Ajusta en base a las necesidades del usuario, a la situación del usuario en ese momento, y a las características del alimento, una variable que servirá de "peso", para mostrar las recomendaciones más adecuadas. No obstante, este algoritmo da mucho peso a las características del alimento y las necesidades del usuario en ese momento, sobre las necesidades del usuario de todo el día. Y si dos alimentos de distinta calidad tienen una puntuación similar, se recomendará el de mejor puntuación sin tener en cuenta la calidad.

\subsection{Adaptación a diferentes plataformas}
Hoy en día las plataformas multimedia y los distintos dispositivos "smart", son cada vez más utilizados en el ámbito de la enseñanza. Siendo una metodología de educación cada vez más aceptada. Dentro de las nuevas tecnologías, las Tablets y los SmartPhones, son las herramientas con mayor crecimiento en cuanto a su uso se refiere; Por ello, la idea de una aplicación interactiva, que enseñe al usuario a habituar un estilo de vida saludable, ha de ser una aplicación que esté al alcance del usuario de la forma más sencilla posible. La ampliación de este proyecto a distintas plataformas que mejores la comunicación con el usuario es en parte una obligación.

