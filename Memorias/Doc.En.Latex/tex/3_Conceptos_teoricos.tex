\capitulo{3}{Conceptos teóricos}

\section{Introducción}
Se explicará todo concepto teórico interno y/o externo, usados en la metodología del programa para su correcto funcionamiento y la coherencia entre funciones, acercándose lo máximo posible a la experiencia real que se quiere conseguir con este proyecto a la hora de interconectar los datos con los métodos, dando así el mayor rigor en cuanto a veracidad.
\section{Nutrición}
La nutrición es uno de los ámbitos principales de estudio a lo largo de este proyecto, para así lograr la máxima autenticidad de los resultados. En cada cálculo y método que se realizan en el programa, se tienen en cuenta una serie de valores y datos meticulosamente estudiados para poder usarlos en el programa.\\

La nutrición como un avance médico, es un tema que a día de hoy esta en pleno auge. Numerosos estudios resaltan el estrecho vínculo entre un buen estilo y una buena calidad de vida en cuanto al término de salud se refiere. Ya no solo en patologías directamente relacionadas con la alimentación, sino en otro tipo de patologías más graves, en las cuales se puede reducir enormemente el riesgo de agravamiento con una buena alimentación.\cite{prevCancer}
\subsection{Reparto calórico}

Aunque durante el periodo de investigación se ha demostrado que todos los nutrientes principales de un alimento son igual de buenos y necesarios, la predisposición del cuerpo ante la comida varía según la etapa del día. \\

Durante el desayuno somos más propensos a asimilar los nutrientes, mientras que a la hora de la cena nuestro metabolismo está más ralentizado y se debe cenar con mayor moderación, pero no es aconsejable restringirlo estrictamente.\\

Para ello se realizó un estudio contrastando diferentes datos y se calculó a través de los datos obtenidos el porcentaje total de cada macronutriente (hidratos, proteínas y grasas) en cada comida.
\subsubsection{Reparto total de los macronutrientes}
Hay que tener en cuenta que este proyecto ha sido diseño para personas con diferentes patologías y que como tal, tendrán diferentes distribuciones. En un principio se pensó en todas las combinaciones posibles: baja en grasa, alta en grasa, baja en hidratos, alta en hidratos, etc.\\

Pero tras hablar con un doctora especialista en Endocrinología del Hospital Universitario de Burgos, María Maravi, me afirmó que: 
\begin{quote}
\textit{"Nosotros no trabajamos con dietas altas o bajas en hidratos, pues son necesarios por igual para todo el mundo y por lo cual, la modificación de este parámetro solo puede traer problemas al paciente. Un diabético ha de tomar una cantidad normal de hidratos."}
\end{quote}
De este modo la especialista nos proporcionó la tabla de variaciones según el tipo de dieta (Tabla 3.1).
\tablaSmall{Distintos tipos de dietas y su distribución}{l c c c c}{distribucionDietas}
{ \multicolumn{1}{l}{Dietas} & Hidratos & Grasa & Proteínas & Ejemplos\\}{ 
Normal & 55\% & 20\% & 25\% & Cualquier persona  \\
Baja en grasa & 55\% & 15\% & 30\% & Problemas cardiovasculares \\
Alta en grasa & 55\% & 25\% & 20\% & fibrosis quistica \\
Baja en proteína & 57\% & 25\% & 23\%  & Insuficiencia renal \\
Alta en proteína & 50\% & 15\% & 35\% & Pacientes en diálisis\\
} 
\subsection{Reparto total calórico}
Internamente se realizan múltiples cálculos, los cuales son necesarios ya sea para la correcta recomendación alimenticia del usuario, como para el correcto tratamiento de los datos. \\

Debido a la diversidad de información encontrada en la web, fue muy complicado llegar a un punto estable donde los datos tuvieran una coherencia para todos y fueran veraces. Para que las cifras cuadrase se creó una  matriz donde se dividieron las columnas por macronutrientes y las filas por el tipo de comida en concreto. Esta matriz debía tener como resultado un porcentaje que cuadrase en varios ejes y un subporcentaje que cuadrase en el total, de esta manera se tenía controlado en todo momento la veracidad de los datos. Como se puede observar en la Tabla 3.2.
\tablaSmall{Distribución calorica a lo largo del día.}{l c c c c}{distribucionDietas}
{ \multicolumn{1}{l}{Comidas} & Grasa & Hidratos & Proteínas & carga TOTAL \\}{ 
Desayuno & 15\% & 30\% & 24\% & 24.75\%  \\
Almuerzo & 10\% & 25\% & 14\% & 13.5\%\\
Comida & 20\% & 25\% & 24\%  & 30.5\% \\
Merienda & 20\% & 10\% & 14\% & 11.5\% \\
Cena & 10\% & 10\% & 24\% & 19.75\% \\
} 
Se estudió para ello la diferente predisposición de las comidas y los nutrientes internos de cada uno.

\subsection{Nutriscore (Algoritmo del semaforo)}
De manera objetiva, se creó un método de medición para la conveniencia de los alimentos (calidad). Para que dicho cálculo mantuviese correlación con la actualidad que cubre el proyecto, se decidió usar el recurso Nutriscore, actualmente usado en España.\\

A continuación, se muestra en imagen el algoritmo seguido de una breve explicación:\\
\imagen{Nutriscore}{Algoritmo Nutriscore}

La imagen anterior muestra el algoritmo , el cual, se basa en el sumatorio de una serie de valores negativos ( Tabla roja), que son: Energía (kilojulios), Azúcares (gramos), Grasas saturadas (gramos), Sodio (miligramos). Estos valores reciben una puntuación del 0 al 10, en base a la cantidad que contenga un alimento por cada 100 gramos de comida de estas características. Y una serie de valores positivos (tabla verde)que son: frutas, fibra y proteínas. Estos valores varían entre 0 y 5, y adquieren esta puntuación de la misma forma que los anteriores.\\

A mayor puntuación, mayor presencia tiene sobre el resultado final, que se haya restando a la puntuación obtenida de las características negativas, la puntuación de las características positivas. En base al resultado de la diferencia se le asigna un color y letra. Como bien se ve en la tabla \ref{tabla:nutri} 
\tablaSmall{Resultado final del Algoritmo Nutriscore}{l c c c c}{nutri}
{ \multicolumn{1}{l}{Puntuación} & Letra & Color \\}{ 
<-1 & A  & Verde oscuro \\
0 a 2 & B & Verde \\
3 a 10 & C  Amarillo  \\
11 a 18 & D & Naranja  \\
>19 & 10\% & E &Rojo  \\
} 
\subsubsection{Imperfecciones}
Este algoritmo contiene fallos, pues se basa única y exclusivamente en cifras, como en las cifras de azúcar, de proteína, de sal, etc. Por lo cual, cosas saludables como puede ser el aceite de oliva virgen extra, tienen como resultado una pésima calidad, mientras que quizás ultraprocesados como los cereales bajos en azúcares, bebidas edulcoradas y cosas de gran similitud nutricional, adquieren una buena calidad. \\
Esto se debe por ejemplo, a que por cada 100 gramos, el aceite de oliva tiene una alto contenido en kilojulios y en grasas, por lo cual lo coloca en la categoría E, cuando debería ser un A.

\section{Prevenciones}
Debido al propósito del proyecto como ayuda a la prevención o avance de enfermedades/patologías por parte del usuario, había que estudiar qué tipo de prevenciones había y donde cabía el proyecto.\cite{prevencion}
Tipos:
\begin{itemize}
\item	Prevención Primaria: Evita la adquisición de la enfermedad. Estas técnicas actúan suprimiendo los factores desfavorables antes de que puedan llegar a general la enfermedad.
\item	Prevención Secundaria: Detectar la enfermedad y evitar su posible progresión. Interviene cuando se inicia la enfermedad, y su función principal es que el diagnóstico y tratamiento precoz mejoren la evolución y control de las enfermedades.
\item	Prevención Terciaria: Comprende aquellas medidas dirigidas al tratamiento y a la rehabilitación, además de la ralentización de su progresión. Interviene cuando las lesiones son irreversibles, y la enfermedad está arraigada, pudiendo llegar a ser crónica.
\end{itemize} 

Análisis:\\
Respecto a la \textbf{prevención primera}, una buena alimentación junto con una serie de hábitos de vida saludables puede impedir enormemente el desarrollo de la enfermedad, siendo la recomendación de primera elección por los profesionales a sus pacientes.\\
En la \textbf{prevención secundaria} también entra esta categoría. En patologías como la diabetes o problemas cardiovasculares, la dieta es vital para su tratamiento. Un buen hábito de vida y de alimentación puede tener incluso mejores resultados que la medicación en algunos casos (como se observa en el estudio de prevención de diabetes \cite{DPP}).
\\
En el caso de la \textbf{prevención terciaria} es más complicado de asegurar. A falta de estudios concluyentes, no se puede afirmar con exactitud que el proyecto cubra este tipo de prevención en estadios tan avanzados de la enfermedad.\\

Lo que si es cierto es que una buena alimentación, junto con actividad física moderada a lo largo de la vida, siempre es recomendable para la salud de cualquier persona.\\
\section{Enseñanza}
La idea principal del proyecto es evitar otra aplicación repetitiva sobre la gestión o impartición de dietas, las cuales suelen ser abandonadas por la mayoría de los españoles al paso del tiempo.\\

Para ello se estudió varios métodos de enseñanza, aprendizaje a través de métodos multimedia, etc. 
\subsection{Métodos multimedia}
En los últimos años, el desarrollo de aplicaciones tanto móviles como de escrito han sido integradas activamente en los diferentes procedimientos de enseñanza. Se considera que las aplicaciones a día de hoy ofrecen muchas ventajas al mundo de la enseñanza:
\begin{itemize}
\item	Comunicación a tiempo real entre el método de enseñanza y el alumno, estando siempre al alcance de éste.
\item	Cómoda distribución de tareas, siendo más fácil el orden y la organización.
\item	Ayuda a la superación de las barreras geográficas.
\end{itemize}

En educación, se ha pronosticado a que lo largo de los años dominará el uso de la Tablet. Por ello, parte de la progresión del proyecto es su extensión a Tablets y Smartphones. \cite{ensenanza}
\subsection{Autoaprendizaje}
Debido a que el uso de este tipo de aplicaciones carece de una figura presenta durante la fase de aprendizaje del usuario, se buscó un método en el que el usuario por su propia índole, aprendiese y poco a poco fuese adecuando su nuevo habito de vida, en base a sus propias elecciones. Esto otorga al usuario versatilidad, dándole la opción en todo momento de elegir que es lo que mas le apeteciese, recomendando la mejor opción y complicando que el usuario hago una mala elección. De esta manera se juega con la idea de hacer creer al usuario que es él el que toma las buenas o malas decisiones.
\section{Autoaprendizaje del valor TMB}
Pese a que en ningún momento se llego a implantar, se desarrolló un estudio sobre inteligencia artificial o redes neuronales en la aplicación para hacer del programa un sistema único.
Llegándose a crear una neurona artificial con la intención de programar una función que fuera la \textit{autoselección} de los menús, pero esto solo reforzaría las elecciones más comunes del usuario que no tienen por qué ser adecuadas.\\

Finalmente, se ideó un sistema de autoaprendizaje basado en el reajuste del valor de una variable, la cual cambia en base a la progresión del usuario. Esto permite tener un sistema único que resuelve el principal problema del calculo del TMB, la generalidad. \\
\clearpage

\section{Métodos de recomendación}
Se meditaron diferentes métodos de recomendación:
\begin{itemize}
\item	Filtro colaborativo basado en modelos: Utilizan los datos para ajustar modelos que después pueden ser utilizados para proponer recomendaciones.
\item	Filtro colaborativo basado en usuario: Es un tipo de filtro basado en memoria, y se basa en recomendar lo que usuarios similares a nuestro usuario han escogido.
\item	“Vecino más cercano”: Su cálculo se basa en la correlación de Pearson, y es muy similar (por no decir idéntico) al filtro colaborativo basado en usuarios.
\end{itemize}

Todos estos métodos son ampliamente utilizados por grandes distribuidores como Amazon, Netflix ...\\
Pero para este proyecto presentaban dos grandes inconvenientes:
\begin{itemize}
\item	Como bien se vió en la asignatura de gestión de la información, son métodos que requieren de muchos usuarios y productos para que funcionen de manera precisa, sino las recomendaciones pueden ser totalmente aleatorias, dando paso a una muy mala recomendación.
\item	La más importante, si este proyecto es creado por el desconocimiento extendido sobre la buena alimentación y sus beneficios, no se puede comparar a un usuario con el resto de ellos, pues a la larga el programa se “estropeará”, y pasará a dividir a las personas en dos grandes grupos: los que se cuidan bien y los que no. Y es precisamente lo que se quiere evitar.
\end{itemize}

Por ello, pese a ser uno de los filtros menos recomendados en la asignatura de Gestión de la Información,  para este proyecto fue el más apropiado. Se pasó a usar un filtro basado en contenidos, por lo general este filtro necesita saber las características del usuario y del contenido de los productos (en este caso alimentos), para poder realizar la recomendación. Esto obviamente tiene un gran coste de información y por ello no es recomendable. Pero en este proyecto se parte con esa información. Básicamente, sabiendo exactamente las características que el usuario busca en esa comida, se le recomienda dicho producto (Menú) que más se adapte a lo que busca.


