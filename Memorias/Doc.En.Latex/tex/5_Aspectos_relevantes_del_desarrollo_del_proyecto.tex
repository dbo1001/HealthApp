\capitulo{5}{Aspectos relevantes del desarrollo del proyecto}

\section{La idea}

Con el objetivo de innovar se crea esta idea propia, buscando la originalidad del proyecto y renovación de los métodos tradicionales. Para ello, se lleva a cabo una investigación sobre las necesidades que sufren las personas actualmente, con las ultimas corrientes de moda y que además pudiera resultar útil. La salud y la buena calidad de vida son temas de vital importancia para las personas, que con el paso del tiempo cobran cada vez más protagonismo.\\

Una vez establecida las bases del proyecto se analizan los posibles métodos u opciones para conseguir el fin del proyecto:\\

\begin{itemize}
\item	Deporte: Se excluye por la inmensa cantidad de aplicaciones de planificación de entrenamiento que existen a día de hoy.
\item	Nutrición: Inicialmente se descartó la idea debido a las mismas razones que lo anterior. Existen multitud de aplicaciones relacionadas con la planificación alimenticia.
\item	Educación nutricional: Enfoque de la nutrición, visto desde el punto de vista del auto-aprendizaje, para usuarios sin experiencia. Se optó por este desarrollo, debido a la falta de plataformas o aplicaciones, cuya finalidad sea inculcar a los usuarios este nuevo habito.

\end{itemize}
Finalmente, la idea consiste en un sistema de recomendación nutricional, con la finalidad del aprendizaje de un hábito de vida saludable.

\section{Formato de trabajo}
Se desarrollo una metodología ágil de forma incremental. Se empezó formalizando las bases hasta llegar a la etapa final. No se pasó a la siguiente etapa sin haber terminado la anterior.\\

Las tres principales etapas del proyecto fueron:
\begin{enumerate}
\item	Funcionamiento por interfaz Shell
\item	Funcionamiento de manera gráfica
\item	Finalización detallada.
\end{enumerate}

De esta forma, se fueron creando todas las características progresivamente. Se priorizó el desarrollo del proyecto frente a la velocidad de dicho desarrollo. Debido al sistema de organización por etapas, se consigue un progreso fluido. Utilizando los recursos substraídos de la etapa anterior, impidiendo parches en el código o incoherencias.\\

Se desarrollo en estas tres etapas para el perfeccionamiento del código. Se desarrolla primero una aplicación por interfaz de comandos, permitió perfeccionar la aplicación, sin tener que preocuparse por otro tipo de problemas. Se crea una estructura sólida, que posteriormente heredaría la fase de la interfaz gráfica.\\

\section{Algoritmos del programa}
Se desarrollan desde cero una serie de algoritmos o sistemas de tratamiento de los datos. Haciendo de esta parte algo propio de este proyecto.
\subsection{Algoritmos de recomendación}
Es el algoritmo que estructura todo el sistema de recomendación, nace de la necesidad de que la recomendación se ajuste al máximo a las necesidades del usuario.\\
Si se ordenaba única y exclusivamente por la calidad, nunca se verían los menús de peor calidad, y el usuario nunca aprendería a llevar un estilo de vida saludable; además de que podría haber una desigualdad inmensa entre las calorías ingeridas y las necesitadas.\\
Se pensó en hacer una formula basada en la diferencia calórica, cosa que se adecuaba más a las necesidades del usuario, pero creaba un desajuste en los macronutrientes, rompiendo la idea de dieta equilibrada. Por ello, se ideo otro prototipo de formula.\\

La fórmula tiene en cuenta la proporción de las necesidades calóricas del usuario, partiendo de cada macronutriente, se calcula la diferencia entre la diferencia de macronutrientes que debería llevar y los que llevo, todo llevo partido por la suma de la diferencia entre: Las kilocalorías que debo comer y las que tiene el alimento y la diferencia entre el macronutriente especifico y lo que el alimento tiene.\\
\begin{equation}
DiferenciaMacronutriente = \frac{(KMCD - A)}{((KCDT-KMM)+(KMTC-KTM))}
\end{equation}
Donde:
\begin{itemize}
\item \textbf{KMCD:} Número de Kcalorias del macronutriente (Hidratos,proteinas o grasas) de la Comida (Desayuno, almuerzo...) que debería llevar en esa comida
\item \textbf{A:} Actual número de kcalorias de ese macronuetriente que llevo en todo el día (Si estoy en Comida: desayuno+almuerzo+comida).
\item \textbf{KCDT:} Kilocalorias que debería comer en esta comida concreta para este macronutriente concreto.
\item \textbf{KMM:} Kilocalorias del macronutriente concreto del menú.
\item \textbf{KMTC:} Kilocalorias totales que debería comer en esta comida.
\item \textbf{ktm:} kilocalorias totales que tiene el menú a recomendar.
\end{itemize}

\subsection{Reajuste del TMB}
Durante el proyecto, se usa la fórmula para el cálculo del Metabolismo Basal, esto da como resultado las calorías que el usuario consume en un día normal. A esta cantidad de calorías se le suele variar una cantidad de calorías que suele oscilar entre las 300 y 500 calorías. Para poder bajar o subir de peso de manera controlada (Aproximadamente 3 kilogramos al mes). Esta fórmula antiguamente estaba muy extendida entre nutricionistas y endocrinos a la hora de diagnosticar/proporcionar una dieta. Con los años, entro en desuso, debido a la generalidad que hacer la formula sobre las calorías que gasta cada persona, sin tener en cuenta metabolismos, posibles enfermedades, etcétera.\\

Desde el principio fue un problema que limitaba las posibilidades del proyecto. Por ello, se desarrolló un método que permite, a través de una coleccióm de registros, llevar la cuenta de las variaciones de peso que sufre el usuario, y amoldar ese valor fijo, permitiendo su reajuste, y creando una fórmula que se adapta con mayor precisión a las necesidades del usuario.\\

\subsection{Persistencia de los datos}
No es propiamente un algoritmo, sino un recurso para solucionar una de las mayores limitaciones que presentaba usar una hoja de cálculo como base de datos y evitar  datos sobre-escritos y diferentes problemas.\\
Se procuró durante todo el programa atomizar al máximo posible cada método sobre la base de datos, haciendo que los cambios sean lo más inconexos y manejables para el usuario. \\

En el método añadir alimento se rompe esta norma, añadiendo un alimento en la base de datos que hay que guardar, pero sin guardar todos los cambios en el resto de los alimentos, por si el usuario no quisiera hacerlo. 
\subsection{Distinción entre tipos de comida}
Uno de los retos a afrontar a lo largo de este proyecto fue el como se iba a indicar que tipo de comida/menú era. Sin duda no fue uno de los retos más complejos abordados a lo largo del proyecto, pero si presentó diferentes técnicas a la hora de abordarlo.\\

Recordamos que un mismo menú, puede pertenecer a varias comidas del día, entonces se pensaron varios métodos hasta llegar al óptimo, los cuales explicaremos con brevedad a continuación: \\

Se realizó un estudio sobre los diferentes métodos de distinción que se podían realizar. Se pensó en cadenas de caracteres (String) separadas por un delimitador cualquiera, y cargarlo en forma de cadena de caracteres separandolo utilizando algún delimitador, pero esto provocó ciertos problemas de carga y almacenamiento, problemas  que aumentaban la complejidad del algoritmo al buscar a través de un bucle todas las posibilidades.\\

Se ideó una concatenación de bits donde 1 es que era esa comida y 0 que no, es decir: 11010, Significa que no es ni comida ni cena. Esto presentaba el mismo problema que la cadena de caracteres y era quizás más confuso para su compresión, Además del pequeño inconveniente de que si un menú no era desayuno (Es decir empezaba en 0), en algunas ocasiones se almacenaba sin ese dígito creando una desigualdad en el programa. Por estas razones se deicidio pasar al siguiente método, una mezcla de ambas, se iba a usar una cadena para calcular un único valor que cargar. Finalmente se pasó esa cadena de bits a un valor numérico, de esta manera se asegura una única carga, y el tratamiento a través de cláusulas condicionales simples.\\

\textbf{Ejemplo:}\\
Un alimento que sea comida y cena: 00101 -> Se representará con el número 5. \\
Esto permite la carga de un único elemento, sin posibilidad de fallo además de un calculo sencillo, cuyo mayor inconveniente es llevar a cabo todas las cláusulas condicionales, para hacer la criba por tipo de comida.\\

\section{Conocimientos aprendidos en la carrera}
\subsection{INTRODUCCIÓN}
Apartado donde se hablará de forma breve sobre todo conocimiento aprendido durante los años de carrera, relacionado directamente con el desarrollo del proyecto.
\subsection{Conocimientos sobre python}
A lo largo de la carrera, se han estudiado diferentes lenguajes de programación (Ej: Java, C, CSHARP...), entre los cuales se encontraba Python. Por ejemplo, en asignaturas como Sistemas Inteligentes, Nuevas tecnologías, etc., o vistas también en la politécnica de Varsovia, en asignaturas como “Algoritmia y computación matemática”.\\

Además, varias de las librerías usadas durante este proyecto fueron utilizadas en dichas asignaturas, haciendo que el proyecto, partiera desde un punto avanzado facilitándome el aprendizaje.
\subsection{SCRUM}
Quizás uno de los aspectos más destacados a la hora de realizar el proyecto, fue la organización por parte del alumno, a través de la metodología SCRUM, que permitía al alumno mantenerse centrado en cada nueva tarea, no pasando a la siguiente hasta dejar esta zanjada del todo. \\

Para la realización del proyecto, se usa una pizarra física, donde se colocan las diferentes tareas, con una fecha límite de realización, de esta manera, toda idea que se planteaba durante el desarrollo era plasmada en la pizarra y posteriormente hecha. \\
La pizarra era dividida en cuatro partes (pequeña adaptación del SCRUM).\\
\begin{enumerate}
\item	\textbf{TODO:} Por hacer, eran todas las tareas, pautas, e ideas que iban surgiendo a lo largo de la semana o tras las reuniones semanales. Había un máximo de cinco, para evitar que el proyecto se atascase por una carga excesiva de trabajo. De TODO se pasaba a in progress. 
\item	\textbf{IN-PROGRESS:} En proceso, máximo dos cosas, una principal y una secundaria, la principal, consistía en el desarrollo de grandes métodos, cálculos, etc. Mientras que los secundarios, podían ser trabajos de investigación, rellenar datos o cualquier detalle que faltase de completas. De esta manera se evitaba la bifurcación del trabajo, haciendo del camino, pequeños objetivos, claros y directos a completar. Cuando el alumno pensaba que había terminado la tarea, la movía a “TO-VERIFY”.
\item	\textbf{TO-VERIFY:} A verificar, es decir, falta la aprobación de los cambios realizados, si eran cambios propuestos en las reuniones semanales, eran zanjados en la siguiente reunión semanal; De esta forma se creaban distintos modelos y versiones de métodos, perfilando al máximo cada detalle en cada método. Las pequeñas tareas, eran resueltas, ya fuera en las reuniones semanales, o preguntando a personas, sobre cómo sería para ellos el correcto funcionamiento del programa.
\item	\textbf{DONE:} Una vez todas las partes se ponían de acuerdo en que la tarea estaba terminada, pasaban a DONE, este apartado existía básicamente, para hacer mella en el trabajo que el alumno realizó durante la semana, haciendo palpable las semanas donde se tuvo una mayor o menor carga de trabajo, pudiendo regular esto, y teniendo de esta manera un avance constante.
\end{enumerate}
En progreso solo se podía encontrar una tarea. Si el tutor catalogaba una tarea, que se encontraba en "TO-VERIFY"; como incompleta pasaba de nuevo a "In progress" de esta forma, semanalmente se cumplían con una serie de propósitos de forma clara y concisa. \\
Este método fue estudiado en la asignatura de gestión de proyectos por el alumno, y culminado en su periodo de prácticas con el instituto de Castilla y León, el cual, usaban de manera muy similar a la adaptación del alumno, La metodología SCRUM para gestionar todos sus proyectos.\\

\subsection{Investigación}
De vital importancia es el nivel de capacidad de investigación y búsqueda de la información La velocidad y capacidad, de saber buscar, es uno de los conocimientos más importantes que se adquieren a lo largo de la carrera. Conocimiento indirecto, pero presente en todas y cada una de las asignaturas que se han cursado, dicho nivel de investigación permite verificar de forma más precisa, tanto los datos o recursos necesarios para una investigación como la solución de errores que puedan emerger durante el desarrollo.
\section{Conocimientos externos a la carrera}
\subsection{INTRODUCCIÓN}
En este apartado, se explicarán los aspectos relevantes del desarrollo del proyecto externos a los conocimientos iniciales del alumno. Todo aquello que ha tenido que aprender para poder llegar al resultado final.
\subsection{Psicología}
Basándose en los principios de la psicología se busca la interacción adecuada con el usuario. Este proyecto no busca una forma rápida de llegar al cliente/usuario, sino constante. El cliente ha de convertir, las pautas de la aplicación, en su propio estilo de vida. Por ello se estudió las distintas formas de enseñanza, así como, el auge de la tecnología en el sistema educativo.
\subsubsection{Aprender a aprender \cite{autoAprendizajeForma}}
Usado en la educación primaria e infantil, es un método para enseñar al alumno, en base a que el alumno encuentre la propia respuesta. De esta manera desarrolla los estímulos positivos del alumno ante la adquisición de nuevos conocimientos. \\
La idea principal nace de las ventajas que tiene implementar un estilo de vida, sobre el seguir órdenes estrictas. Tras un periodo de investigación, se decantó por un estilo progresivo, en el que se intenta concienciar al usuario de cada decisión que toma, haciendo que el aprendizaje se base en pequeñas metas personales, y en la concienciación del usuario para que, él mismo, se dé cuenta de que es lo mejor para él y aunque de manera lenta pero segura, llegue a la meta que tenga como objetivo.
\subsection{Nutrición}
Conocimientos básicos sobre el mundo de la nutrición, el reparto da los distintos nutrientes de cada alimento a lo largo del día y la correcta distribución de las calorías en base a las características del usuario.\\

Para el desarrollo del proyecto, se ha necesitado conocimientos sobre la calidad de los alimentos, así como, la predisposición de cada tipo de patología en base a las características de los alimentos. Logrando, a través del estudio e investigación, un análisis detallado para la veracidad de los datos a tratar.
\subsubsection{Calidad de los alimentos}
Parte fundamental del proyecto, la calidad marca el umbral por el cual se criban los alimentos. Este umbral es usado para el sistema de recomendaciones. Entre las posibles recomendaciones se encuentran las de umbral más bajo, complicando la mala elección por parte del usuario.\\
La calidad se mide con el \textbf{algoritmo Nutriscore}:\\
El algoritmo del semáforo o NUTRISCORE se basa en una serie de características buenas y malas de los alimentos, y según el resultado de la diferencia, la cual, se acaba cribando en uno de los cinco posibles valores: A, B, C, D Y E, ordenador de mejor a peor.\\

\subsection{Interfaces Gráficas}
Pese a tener los conocimientos sobre programación en Python, se desconocía cualquier tipo de implementación de interfaz gráfica. Debido a esto, gran parte del proceso fue la selección de la librería que se utilizaría durante el resto del proyecto para la implementación de este.\\
Finalmente se escogió Tkinter, por su sencillez, pues permitió dar grandes avances y adaptarse a este nuevo sistema rápidamente. Permitiendo que la interfaz se complicará de manera exponencial, aplicando todo aquello que se iba aprendiendo de manera colateral a lo ya existente.
\section{Fases del proyecto}
Las fases de este proyecto pueden ser divididas de diferentes modos, según qué criterio se utilice, de por si podemos dividir todo el proyecto en tres grandes fases: Idea, desarrollo y testing. Teniendo en cuenta esto, hay que tener en cuenta que el desarrollo a su vez se dividió en otras tres fases principales que fueron: creación de estructuras iniciales, desarrollo en interfaz de comandos y desarrollo en interfaz gráfica. 
Por lo cual un breve esquema del desarrollo podría ser:
\begin{itemize}
\item Nacimiento y pulimento de la idea
\item Desarrollo:
\begin{itemize}
\item Creación de la estructura de los datos.
\item Carga y manejo de los datos
\item Creación del proyecto por interfaz de comando
\item Traducción del proyecto a interfaz gráfica
\item Pulimento de detalles y nuevas funcionalidades
\end{itemize}
\item Pruebas
\item Desarrollo de la documentación pertinente
\end{itemize}
\subsection{Cronograma: }
\imagen{cronograma}{Tabla cronograma al completo}